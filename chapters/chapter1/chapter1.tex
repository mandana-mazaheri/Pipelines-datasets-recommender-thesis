\chapter{Introduction}

\label{introduction}

Some parts of this section are derived from the definitions provided at \url{https://conp.ca/about-the-conp/} \TG{It's weird to start with that. You should refer to the website as a reference and cite it as other references.}

%what is open science?
Traditionally, scientific research was conducted in the research or academic institutions and was kept there. That was Closed Science \TG{remove capitals, closed science isn't a concept}, where after the findings were published, the data were often inaccessible to others due to privacy issues and ownership of the resources. This made data difficult to find and access and made further research on that subject slower. For such reasons, Open Science emerged to prepare and release the data to the whole scientific community worldwide, so they can collaborate in research and discovery, which leads to faster growth of findings.

Open Science is an umbrella term covering open dissemination of data, manuscripts, software, materials, methodologies, and other outputs that scientists produce in their research. Open Science also aims to make the scientific process more transparent and accessible. Open Science research allows others to collaborate and contribute to the study using freely available research data and all the resources in the research process, which facilitates the reuse, redistribution and reproduction of the research outcome and its underlying data and methods~\cite{bartling2014opening}. \TG{Make sure that all references are complete (authors, title, date, journal or conference name, pages, volume/number for journal papers. }

%why open science is important?
In Open science, it is essential to share the resources correctly and make the research findings reproducible, since due to shortcomings in many current methods for sharing and capturing data, ``approximately 50\% of all research data and experiments are considered not reproducible, and the vast majority (likely over 80\%) of data never makes it to a trusted and sustainable repository" ~\cite{ayris2016realising}.

Moreover, Open Science is increasingly important in the current science world, beneficial for scientists, patients, and the public and has emerged as a framework to improve the quality of scientific analyses. There are several reasons why Open Science is necessary. The findings and output of publicly-funded scientific research would be more available and accessible. Therefore the scientific works would be more transparent and reproducible. Also, the public implication would be possible in conducting the research and might impact the results. Also, open peer-review would be possible, encouraging broader and more transparent review processes by extending knowledge exchange between researchers~\cite{wolfram2020open,ross2019guidelines}. 


% how open science is defined? Any other definition that FAIR?
There are different definitions for the concerns and principles required to follow in Open Science. As mentioned above, Open Science encompasses a variety of areas, including open access to publications, open research data, open-source software/tools, open workflows, open educational resources, and alternative methods for research evaluation, including open peer-review~\cite{pontika2015fostering}. There is a set of five broad concerns as ``schools of thought"; defined by Fecher \& Friesike in 2014~\cite{fecher2014open}, to be considered to implement these practices.

These concerns are represented in Figure~\ref{fig:5school} \TG{broken reference}: \TG{fix quotation marks in the following sentences and everywhere in the thesis: ``this sentence has correct quotation marks"} 'Democratic school' believes scholarly knowledge (including publications and data) should be available freely for all. The second one is "Pragmatic school," which aims to make scholarly methods transparent and concerns with efficient knowledge creation through collaboration and critique. The third is 'Infrastructure school', mentioning "efficient research requires readily available platforms, tools and services for dissemination and collaboration "\href{https://open-science-training-handbook.gitbook.io/book/open-science-basics/open-concepts-and-principles}{from here} \TG{don't use hyperlinks, use references instead (everywhere in the thesis)}. The fourth is 'Public school', claiming that the public should collaborate in research and that the scholarship should be more readily understandable through less formal communicative methods. The last one is 'Measurement school' that believes it is required to define alternative metrics to track and measure the impact of scholarship.

\begin{figure*}[ht]
  \centering
  \includegraphics[width=\textwidth]{figures/5schools.png}
  \caption{Five Open Science Schools of Thought. Figure reproduced from~\cite{fecher2014open} }
  \label{fig:5schools}
\end{figure*} 


Wilkinson defines the following widely referred definition for principles of Open Science in 2016 as FAIR principles~\cite{wilkinson2016fair} which claims that the research should be Findable, Accessible, Interoperable, and Reusable for all users. Many of the currently available platforms for sharing resources are trying to guarantee the FAIR principles. Each one of the principles in FAIR is explained in the background chapter of this thesis.  

%open science in neuroscience, meaning
Open Science affected almost all scientific research studies, particularly neuroscience, our main application domain of interest. Open science practices can help to address many challenges in neuroscience. For instance, the answer to many key neuroscience questions can be found when the research findings and outputs are openly shared. In this case, people can also collaborate and help discover innovative solutions for the treatment of brain-related illnesses. 

%data sharing in neuroscience
The resources to be shared in neuroscience include the datasets consisting of raw data, pre-processed data or the result of analysis applied to raw or pre-processed data, which should be shared on data sharing platforms. The data-sharing platforms in neuroscience and other health-related areas are mostly following the principles defined by the National Institutes of Health (NIH) available~\href{https://grants.nih.gov/grants/policy/data_sharing/data_sharing_guidance.htm}{here}. According to these principles, "Data should be made as widely and freely available as possible while safeguarding the privacy of participants, and protecting confidential and proprietary data". \href{https://loris.ca/index.html}{LORIS} is an open-source framework for storing, processing and sharing behavioural, clinical, neuroimaging and genetic data. Also \href{https://www.braincode.ca/}{Brain-CODE} is "A Secure Neuroinformatics Platform for Management, Federation, Sharing and Analysis of Multi-Dimensional Neuroscience Data"~\cite{vaccarino2018brain}.


% tools are required to process the data, tool sharing in neuroscience 

The data itself is not enough in neuroscience research studies; there are a set of neuroimaging tools and pipelines for processing the data that should be shared as well. Therefore, to guarantee open science principles in neuroscience, it is required to use platforms for sharing neuroscience datasets and neuroimaging tools and pipelines and the possible outcome of conducted analysis. There are many neuroscience data and tool sharing platforms such as \href{https://canadianbrain.ca/mission-vision/}{Canadian Brain Research Strategy (CBRS)}, \href{https://www.nitrc.org/}{NeuroImaging Tools \& Resources Collaboratory (NITRC)}, \href{https://openneuro.org/}{OpenNeuro} and \href{https://conp.ca/}{Canadian Open Neuroscience Platfrom}. Three of these platforms are explained in this thesis NITRC and OpenNeuro in Chapter 2 and CONP in Chapter 3.



%the problem we focused on?
Although there are many platforms for data and tool sharing in neuroscience, it is still required to have a system to help users bridge the datasets and tools together when they conduct an analysis. In the best scenario, the users can see the outcome of analysis applied on data in current platforms. However, providing a list of possible tools/pipelines for a data/dataset will facilitate the user's analysis process and prevent the confusing and time-consuming process of selecting an appropriate tool/pipeline for the candidate data/dataset. The same scenario would be helpful when the user needs to select data/dataset to execute a candidate tool/pipeline.

This thesis aims to investigate the feasibility of implementing a recommender system to recommend appropriate tools/pipelines and data/datasets given the other based on the available records from previous executions. Consistently with our motivating use case, we focused on the available neuroscience tools/pipelines and data/datasets (from now on 'pipeline' and 'dataset') in the Canadian Open Neuroscience Platform (CONP) available at \url{https://portal.conp.ca/index}. The pipelines in CONP are described in Boutiques~\cite{glatard2018boutiques} which is a software library for sharing tools according to the FAIR principles. CONP has distributed datasets in neuroimaging, transcriptomics, genomics, and other related data modalities. More details about the CONP and its available pipelines and the dataset are provided in Chapter 3.



Recommender systems are widely used and increasingly successful in satisfying the users by suggesting the items they might like and helping them select the better choices faster. There have been many successful works on recommender systems such as Netflix~\cite{bennett2007netflix} for movie recommender and Amazon~\cite{7927889} as product recommenders. There are two main strategies for recommender systems. The first one is Collaborative Filtering~\cite{rajaraman2011mining} which proposes recommendations based on the user-item interactions, recommending an item to a user if similar users like it. The other approach is Content-based Filtering~\cite{pazzani2007content} which recommends a user items that are similar to their previous choices.  

We focused on collaborative filtering for our project to recommend neuroimaging data/datasets and tools/pipelines given the other. In this case, we have a database of the previously executed pipeline-dataset pairs and apply a collaborative filtering model to get the recommended items. We did not select a content-based approach for the recommender system since, in this case, there should be accurate and comprehensive descriptions for all datasets and pipelines while these descriptions are manually written by some colleagues in CONP and are not reliable and consistent.   

 

% what will be in this thesis?

This thesis is organized as follows. In Chapter 2, we explain the details about the required components and background knowledge. Chapter 3 explains the Canadian Open Neuroscience Platform (CONP). Chapter 4 presents the recommender system proposed in this thesis and is submitted as a conference paper to \href{https://supercomputing.org/}{The International Conference for High-Performance Computing, Networking, Storage, and Analysis} \TG{Mention the WORKS workshop}. Finally, Chapter 5 expands the conclusions and explains more about the possible future works on this project.

\TG{The intro is ok, but you should mention the goal of the thesis earlier on. There's a risk that reviewers don't understand where you're going. }


