\chapter{Introduction}
\label{introduction}

In this chapter we explain what Open Science is and how it can be helpful for researchers and scientists in section 1.1 Then in section 1.2 we will focus on our thesis contributions and organizations in 1.3 and 1.4.


\section{Open Science}

Open science is a collection of actions designed to make scientific processes more transparent and results more accessible. Its goal is to build a more replicable and robust science~\cite{spellman_gilbert_corker_2017}. 
Many journals and funding agencies now encourage, require, or reward some open science practices, including pre-registration, providing full materials, posting data, distinguishing between exploratory and confirmatory analyses, and running replication studies. Individuals can
practice and promote open science in their many roles as researchers, authors, reviewers, editors, teachers.

In the world of neuroscience a lot of attempts have been
made to simplify reproducible research, specifically in neuroimaging,
and functional Magnetic Resonance Imaging (fMRI)~\cite{Schöttner_2020}. 

\section{Our approach}

expanded intro:
Specific to our problem! (after one page)


Open science, sharing open science..
many things are shared but not alot to help people
  