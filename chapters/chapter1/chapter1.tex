\chapter{Introduction}
\label{introduction}


Open Science is increasingly important in current science world and has emerged as a framework to improve the quality of scientific analysis. As an open science study, the resources and the analysis should guarantee the FAIR~\cite{wilkinson2016fair} principles to be Findable, Accessible, Interoperable, and Reusable for the scientists and the researchers. In neuroscience studies the resources are including datasets, neuroimaging pipelines and tools, and analysis applied on the data. As a result, many platforms have emerged such as OpenNeuro, NeuroImaging Tools \& Resources Collaboratory (NITRC) and Canadian Open Neuroscience Platform (CONP) to facilitate data and tool sharing.

Although there are many platforms for data and tool sharing in neuroscience, there is still lacking of a system to help users  bridging the datasets and tools together which is the goal of this thesis. To help the users with such system we will adopt an approach based on recommender systems since there have been many successful works on recommender systems in past 15 years such as Netflix~\cite{bennett2007netflix} and Amazon~\cite{7927889}. There are two main strategies for recommender systems, Collaborative Filtering~\cite{rajaraman2011mining} and Content-based Filtering~\cite{pazzani2007content}. We selected collaborative filtering to implement a system to recommend datasets to pipelines and pipelines to datasets in the field of neuroimaging. Consistently with our motivating use case, we focused on the available neuroscience pipelines and datasets in CONP. 


This thesis is organized as follows, in chapter two we explain the details about the required components and background knowledge, chapter three represents our approach, and in chapter four we expand the conclusion. 


