\chapter{conclusion}
\label{conclusion}

In this thesis we ...


expand conclusion 2-3 pages + future work

the execution records are available at the portal
a mock up for the recommender is available at... (will be developed soon)

more data would be helpful...

it will results in using implicit feedback....

for the mockup:
I am integrating this to the CONP, it will provide a full operational system based on ..... validations


limitations point to future works

Collaborative filtering predicts the execution outcome of a given pipeline
on a given dataset with usable accuracy (AUC=0.83) in the context of the
Canadian Open Neuroscience Platform. The performance achieved by our system
outperforms
human expert recommendations, presumably due to syntactical and
infrastructural factors neglected by human experts. 
Future work will focus on the deployment of such a system in production conditions, which will require dealing with less reliable provenance records.

% in which the recommended list of datasets or pipelines will be displayed on the CONP portal interface and available for the users, to see the mock-up for recommender interface on the CONP portal \href{https://github.com/mandana-mazaheri/Pipelines-datasets-recommender-paper/blob/master/result/mockup_for_recommender_on_portal.png}{ click here}. 


The framework could be extended by considering pipelines and datasets 
at a finer granularity. Pipelines can often be used in different ways 
depending on their parametrization. Different parametrizations could 
be identified in the provenance records and recommended accordingly 
for specific datasets. Besides, datasets often consist of multiple 
sub-parts corresponding to different subjects or data types. A recommender 
system could be designed to recommend analyses for such sub-parts, 
resulting in more specific recommendations.