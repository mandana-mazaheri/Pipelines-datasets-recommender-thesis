\chapter{Conclusion}
\label{conclusion}

We designed a model of the Linux page cache and implemented it in the
\simgrid-based \wrench simulation framework to simulate the execution
of distributed applications.
Evaluation results show that our model improves simulation accuracy
substantially, reducing absolute relative simulation errors by up to
9$\times$ (see results of the single-threaded experiment). The
availability of asymmetrical disk bandwidths in the forthcoming
\simgrid release will further improve these results.
Our page cache model is publicly available in the \wrench GitHub
repository.

Page cache simulation can be instrumental in a number of studies. For
instance, it is now common for HPC clusters to run applications in
Linux control groups (cgroups), where resource consumption is limited,
including memory and therefore page cache usage. Using our simulator,
it would be possible to study the interaction between memory allocation
and I/O performance, for instance to improve scheduling algorithms or
avoid page cache starvation~\cite{zhuang2017}. Our simulator could also
be leveraged to evaluate solutions that reduce the impact of network
file transfers on distributed applications, such as burst
buffers~\cite{ferreiradasilva-fgcs-bb-2019}, hierarchical file
systems~\cite{islam2015triple}, active storage~\cite{5496981}, or
specific hardware architectures~\cite{hayot2020performance}. 

Not all I/O behaviors are captured by currently available simulation models,
including the one developed in this work, 
which could substantially limit the accuracy of simulations.
Deeper investigation in cache eviction and data flushing mechanisms 
could help improve the accuracy of the model. 
Relevant extensions to this work include more
accurate descriptions of anonymous memory usage in applications, 
which strongly affects I/O times through writeback cache. 
File access patterns might also be worth including in the simulation models,
as they directly affect page cache content.
Apart from local and network filesystems considered in this work, 
simulating page cache with Lustre filesystem would also be a possible 
extension of the model. 
Another interesting challenge should be simulating big data applications 
on big data frameworks such as Spark or Hadoop.
% readahead and persistent storage could also be adeed
% Simulation results could be made even more accurate by a deeper
% investigation of page cache flushing and eviction order. 

All the results of this work have been collected and made available at the Github repository 
\url{https://github.com/big-data-lab-team/paper-io-simulation} 
with scripts and Jupyter notebooks to generate and view figures. 
Chapter~\ref{method} and Chapter~\ref{experiment} of this thesis have been 
published as an arXiv pre-print:

\begin{quote}
    Hoang-Dung Do, Valerie Hayot-Sasson, Rafael Ferreira da Silva,
    Christopher Steele, Henri Casanova, Tristan Glatard, ``Modeling the
    Linux page cache for accurate simulation of data-intensive
    applications", \href{https://arxiv.org/abs/2101.01335}{arXiv:2101.01335}. 
\end{quote}